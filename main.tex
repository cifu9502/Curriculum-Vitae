%%%%%%%%%%%%%%%%%%%%%%%%%%%%%%%%%%%%%%%%%
% "ModernCV" CV and Cover Letter
% LaTeX Template
% Version 1.11 (19/6/14)
%
% This template has been downloaded from:
% http://www.LaTeXTemplates.com
%
% Original author:
% Xavier Danaux (xdanaux@gmail.com)
%
% License:
% CC BY-NC-SA 3.0 (http://creativecommons.org/licenses/by-nc-sa/3.0/)
%
% Important note:
% This template requires the moderncv.cls and .sty files to be in the same 
% directory as this .tex file. These files provide the resume style and themes 
% used for structuring the document.
%
%%%%%%%%%%%%%%%%%%%%%%%%%%%%%%%%%%%%%%%%%

%----------------------------------------------------------------------------------------
%	PACKAGES AND OTHER DOCUMENT CONFIGURATIONS
%----------------------------------------------------------------------------------------

\documentclass[11pt,a4paper,sans]{moderncv} % Font sizes: 10, 11, or 12; paper sizes: a4paper, letterpaper, a5paper, legalpaper, executivepaper or landscape; font families: sans or roman

\moderncvstyle{classic} % CV theme - options include: 'casual' (default), 'classic', 'oldstyle' and 'banking'
\moderncvcolor{grey} % CV color - options include: 'blue' (default), 'orange', 'green', 'red', 'purple', 'grey' and 'black'

\usepackage{lipsum} % Used for inserting dummy 'Lorem ipsum' text into the template

\usepackage{multicol}

\usepackage[scale=0.75,top=2cm,bottom=3.1cm]{geometry} % Reduce document margins
\setlength{\hintscolumnwidth}{2.3cm} % Uncomment to change the width of the dates column
%\setlength{\makecvtitlenamewidth}{10cm} % For the 'classic' style, uncomment to adjust the width of the space allocated to your name




%----------------------------------------------------------------------------------------
%	NAME AND CONTACT INFORMATION SECTION
%----------------------------------------------------------------------------------------

\firstname{{\huge Jes\'us David}} % Your first name
\familyname{{\huge Cifuentes Pardo} } % Your last name

% All information in this block is optional, comment out any lines you don't need
\title{Curriculum Vitae  }
\address{Rua Alvarenga 2392}{S\~ao Paulo, Brazil}


\mobile{  +55 (11)963241631}
\email{jesuscif@if.usp.br}
%\photo[65pt][1pt]{foto2.JPG}  

%----------------------------------------------------------------------------------------

\begin{document}


\makecvtitle % Print the CV title

%----------------------------------------------------------------------------------------
%	EDUCATION SECTION
%----------------------------------------------------------------------------------------


%\begin{picture}(0,0)
%\put(300,10){\includegraphics[scale=0.25]{foto.jpg}}
%\end{picture}

I am currently a Master's student at the Institute of Physics in the University of S\~ao Paulo (IFUSP). I work at the department of Theoretical Condensed Matter Physics with my advisor Luis Greg\'orio Dias. Before that, I obtained a double-degree in Physics and Mathematics at Universidad de los Andes, Colombia (2016). \\ 
\\
My research interests include: 

\begin{itemize}
    \item Quantum Computation.
    \item Topological Materials.
    \item "Majorana Physics": Search, braiding, topological quantum computation.
    \item Silicon-based quantum computing.
    \item Electronic transport in low-dimensional systems.
\end{itemize}


\section{Education}

\cventry{March 2017 -- March 2019 }{M.Sc. in Physics}{Institute of Physics, Universidade de S\~ao Paulo (IFUSP)}{S\~ao Paulo, SP}{Brasil}{}

\cventry{Jan 2013 -- April 2017}{B.Sc. in Mathematics}{Universidad de los Andes}{Bogot\'a}{Colombia}{GPA: 4.45/5.00 . Class rank \#2 out of 20. Statistics from :\\ \footnotesize{\url{https://planeacion.uniandes.edu.co/dmdocuments/Boletin_estadistico_2016.pdf}} }

\cventry{Jan 2012 -- Oct 2016}{B.Sc. in Physics}{Universidad de los Andes}{Bogot\'a}{Colombia}{GPA: 4.45/ 5.00. Class rank \#2 out of 20}

\cventry{Dec 2011}{High School Diploma}{Colegio Corazonista}{Bogot\'a}{Colombia}{Graduated with honors.}

\section{Research}

\cventry{March 2017 -- March 2019  }{Master's Research Project: "Kondo-Majorana Coupling in Double Quantum Dots "}{Advisor: Luis Greg\'orio Dias}{}{}{Performed a theoretical and numerical transport study of a Double Quantum Dot system coupled to a Majorana chain. Both interacting and non-interacting regimes are considered. Majorana manipulation is achieved by tuning the QD gate voltages and tunnel couplings. The study has potential applications on topological quantum computing architectures. \\ {\footnotesize Draft for Publication: \url{https://git.io/fp2zX}} \hspace{2cm} {\footnotesize Poster: \url{https://git.io/fp2vS}  } }



\cventry{Aug 2016 - Dec 2016}{Undergraduate thesis in Mathematics: "On Gauging Symmetries of 2D Topological Quantum Field Theories" }{Advisor: C\'esar Galindo}{}{}{The algebraic structure used to describe the time evolution of a 1D quantum system is a 2D topological quantum field theory (2TQFT). This structure has acquired some interest because of its relation with topological quantum computation. My thesis presented an algebraic classification of the symmetries of 2TQFTs.  \\ {\footnotesize Link: \url{https://repositorio.uniandes.edu.co/handle/1992/15220} \hspace{1cm} Grade: 5.00/5.00.}}

\cventry{Sept 2015 - May 2016}{Undergraduate thesis in Physics: "Understanding Topological order in 1D fermion chains"}{Advisor: Andr\'es Reyes Lega}{}{}{Studied the topological features of the Kitaev chain.  The analysis allowed us to understand the concept of Topological order as well as the appearance of majorana modes, which are of great interest for quantum computation. \\ {\footnotesize Link: \url{https://repositorio.uniandes.edu.co/handle/1992/17840} \hspace{1cm} Grade: 4.80/5.00.} }{}


%\section{Research Interests}

%\begin{multicols}{2}
%\cvlistitem{Quantum Computation}
%\cvlistitem{Topological States of Matter}

%\cvlistitem{Quantum Information Theory}
%\end{multicols}


%\begin{footnotesize}
%Important courses and seminaries:
%\begin{enumerate}
%\item Workshop on Topological States of Matter
%\item Workshop on Topological Quantum Matter from Theory to Applications
%\item Course in Information Theory for Physicist
%\end{enumerate}
%\end{footnotesize}


\newpage
\section{Awards}
\cvitem{March 2017 -- March 2019}{\textbf{Scholarship(CNPq)} for Master studies in Physics at the University of S\~ao Paulo.}

\cvitem{May 2012}{First place at \textbf{National Undergraduate Math Olympiad},  Colombia.}
\cvitem{July 2011}{Participant of the \textbf{52$^{th}$ International Mathematical Olympiad (IMO)}, Netherlands.  }
%\cvitem{2011}{First place at \textbf{National Physics Olympiad} (High school),  Colombia}



\vspace{2mm}




\section{Events}

%\cventry{May 2018}{ Autumn Meeting of the Brazilian Physical Society}{Foz do Igu\c{c} }{Brasil}{Poster in: Edge-States and Topological Phase Transitions in the Kitaev Chain}

\cventry{Aug 2018}{\mdseries Workshop on Strong Electron Correlations in Quantum Materials: Inhomogeneities, Frustration, and Topology}{ICTP-SAIFR}{S\~ao Paulo}{Brasil}{\textbf{Poster presentation}: Manipulation of Majorana Modes in Double Quantum Dots.}

\cventry{May 2018}{\mdseries Autumn Meeting}{ Brazilian Physical Society}{Foz de Igua\c{c}u}{Brazil}{\textbf{Poster presentation}: Manipulation of Majorana Modes in Double Quantum Dots.}

\cventry{July 2017}{\mdseries 1st Workshop on Topological Quantum Phenomena and Quantum Information Science}{International Institute of Physics}{Sao Carlos}{Brasil}{\textbf{Poster presentation}: Edge-States and Topological Phase Transitions in the Kitaev Chain.}

\cventry{March 2017}{\mdseries Topological States of Matter}{International Institute of Physics}{Natal}{Brasil}{\textbf{Poster presentation}: Edge-States and Topological Phase Transitions in the Kitaev Chain.}

\cventry{July 2016}{\mdseries Journeys on Theoretical Physics}{ICTP-Perimeter-SAIFR}{Sao Paulo}{Brasil.}{}

\cventry{June 2016}{\mdseries Non-Commutative Geometry}{Universidad Javeriana}{Villa de Leyva}{Colombia.}{}

\cventry{May 2016}{\mdseries Random Geometries}{8th School on Mathematical Physics}{Universidad de los Andes}{Bogot\'a , Colombia.}{}{}

\cventry{July 2015}{\mdseries Geometrical, Algebraical and Topological Methods for Quantum Field Theory}{Universidad de los Andes}{Villa de Leyva}{Colombia.}{}

\cventry{June 2015}{\mdseries Workshop on Mathematical Structure and Foundations of Quantum Physics}{}{Universidad de los Andes}{Bogot\'a , Colombia}{\textbf{Talk presentation}: "The GNS construction: An algebraic Formalism for Quantum Physics".}{}

\cventry{May 2015}{\mdseries Topological Quantum Matter: From Theory to Application}{7th School on Mathematical Physics}{Universidad de los Andes}{Bogot\'a , Colombia.}{}{}

\cventry{Jan 2015 - May 2015}{\mdseries Quantum Computing Seminar}{}{Universidad de los Andes}{Bogot\'a , Colombia.}{}{}
%----------------------------------------------------------------------------------------
%	WORK EXPERIENCE SECTION
%----------------------------------------------------------------------------------------

\section{Teaching Experience}
\cventry{2014, 2016}{Teaching Assistant of Linear Algebra I}{Department of Mathematics}{Universidad de los Andes}{Bogot\'a}{Semi-annual part-time job.}{}
%\cventry{2015}{Co-founder of the Student Seminar on Theoretical Physics}{Universidad de los Andes}{Bogot\'a}{}{}
\cventry{2013, 2015}{Teaching Practice in Physics}{Department of Physics}{Universidad de los Andes}{Bogot\'a.}{Semi-annual  part-time job.}{}
\cventry{June 2012, 2013, 2014}{Instructor at Summer Intership in High School Mathematical Olympiads}{Olimpiadas Colombianas de Matem\'aticas}{Universidad Antonio Nari\~no}{Bogot\'a.}{ Summer full-time job. }


\section{Volunteer Work}
\cventry{Aug 2015 -- Nov 2016}{Organizer of the Physics Seminar for Students}{}{Universidad de los Andes}{Bogot\'a.}{Annual part-time.}{}
\cventry{Aug 2016 -- Sept 2016}{Peace lectures on Public Space}{}{Universidad de los Andes}{Bogot\'a}{Part-time.}{}
\cventry{July 2013}{ Leading Guide at the 54th International Mathematical Olympiad }{Olimpiadas Colombianas de Matem\'aticas}{Santa Martha}{Colombia}{July 18 - July 28 . Full-time.}{}



%----------------------------------------------------------------------------------------
%	AWARDS SECTION
%----------------------------------------------------------------------------------------


%----------------------------------------------------------------------------------------
%	COMPUTER SKILLS SECTION
%----------------------------------------------------------------------------------------

%\section{Skills}
%\vspace{1mm}

%\subsection{Physics tests}
%           \cvitem{GRE Subject in Physics}{ Scaled score: 820 , Percentile: 74}
 %          \cvitem{EUF}{Score: 7.9 , Classification: 4 }
%\vspace{1mm}

\section{Languages}

           \cvitem{Spanish}{Mother tongue}
           \cvitem{English}{Fluent. Toefl iBT: 101.  {\footnotesize Reading: 28 , Listening: 30, Speaking: 22 , Writing: 21}} 
           \cvitem{Portuguese}{Fluent}
           


\section{Computational Skills}

\begin{multicols}{2}

\subsection{Programming Languages}
\cvitem{Advanced:}{ C++ , Python, Latex  }
\cvitem{Intermediate:}{Java, R , Matlab}



\subsection{Operating Systems}
\cvitem{Advanced:}{ Ubuntu, Windows }

\end{multicols}
\subsection{Computational Experience }
\cventry{March 2017- December 2018}{NRG simulations of double quantum dots coupled to Majorana wires}{Institute of Physics (USP)}{S\~ao Paulo}{Brasil}{Github: \url{ https://github.com/cifu9502/nrgcode}}



\subsection{Courses}
\cventry{April 2018}{Mini-Course: Tensor Networks and Applications}{Institute of Physics USP}{S\~ao Paulo}{Brasil}{}
\cventry{September 2017}{Minicourse on Machine Learning for Many-Body Physics}{ICTP-SAIFR}{S\~ao Paulo}{Brasil}{}






%----------------------------------------------------------------------------------------
%	INTERESTS SECTION
%----------------------------------------------------------------------------------------

%\newpage


\section{Academic References}

\begin{cvcolumns}
\cvcolumn{Masters Advisor}{Luis Greg\'orio Dias  \\
Universidade de S\~ao Paulo (USP)\\
Institute of Physics  \\
luisdias@if.usp.br \\
\url{http://www.fmt.if.usp.br/~luisdias/} \\
}
  \cvcolumn{Advisor in Mathematics}{C\'esar Galindo  \\
Universidad de Los Andes \\
Department of Mathematics \\
E-mail: cn.galindo1116@uniandes.edu.co \\
\url{https://sites.google.com/site/neyitgalindo/}
}
  \end{cvcolumns}
  
  \begin{cvcolumns}
\cvcolumn{Advisor in Physics}{  Andr\'es Fernando Reyes Lega  \\
Universidad de Los Andes\\
Department of Physics \\
E-mail: anreyes@uniandes.edu.co\\
\url{http://fisica.uniandes.edu.co/profesores/anreyes/students/} \\
}
  \cvcolumn{}{}
  \end{cvcolumns}
\vspace{2mm}





%----------------------------------------------------------------------------------------
%	COVER LETTER
%----------------------------------------------------------------------------------------

% To remove the cover letter, comment out this entire block

%\clearpage

%\recipient{Organizing Comitee}{Villa de Leyva Summer School 2015} % Letter recipient
%\date{\today} % Letter date
%\opening{Dear Organizing Comitee,} % Opening greeting
%\closing{Sincerely,} % Closing phrase
%\enclosure[Attached]{curriculum vit\ae{}} % List of enclosed documents

%\makelettertitle % Print letter title

%\begin{small}

%\end{small}

%\makeletterclosing % Print letter signature

%----------------------------------------------------------------------------------------

\end{document}